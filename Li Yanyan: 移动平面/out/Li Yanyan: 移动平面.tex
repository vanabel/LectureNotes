\documentclass[10pt]{amsart}
% generated by Madoko, version 1.2.0
%mdk-data-line={1}


\usepackage[heading-base={2},section-num={False},bib-label={hide},fontspec={True}]{madoko2}


\begin{document}



\mdxtitleblockstart{}
\mdxtitle{移动平面法}%mdk
\mdxauthorstart{}
\mdxauthorname{Li Yanyan}%mdk
\mdxauthorend\mdtitleauthorrunning{}{}\mdxtitleblockend%mdk

\noindent该笔记是基于Li Yanyan教授在南京大学的报告整理而得.%mdk

\section{1.\hspace*{0.5em}一维情形}\label{section}%mdk%mdk

\noindent假设$u=u(x)$是$(-1,1)$上的函数, 满足如下微分方程%mdk

\[\begin{cases}
-u''=f(u),&x\in(-1,1),\\
u>0,& x\in(-1,1),\\
u(-1)=u(1)=0.
\end{cases}
\]%mdk%mdk

其中, $u\in C^2((-1,1))\cap C^0([-1,1])$, $f\in [0,+\infty)\to \mathbb{R}$ 是Lipschitz连续的. 即, 存在$b\in\mathbb{R}$, 使得
$\left\lvert \frac{f(s)-f(t)}{s-t}\right\rvert\leq b,$
对任意的$s,t\in\mathbb{R}$, $s\neq t$都成立.%mdk

那么, 对任意的$x\in(-1,1)$都有$u(-x)=u(x)$, 且对任意的$0<x<1$有$u'(x)<0$.%mdk

大家可以将其作为一个练习来证明.%mdk

\begin{mdbmarginx}{1ex}{0pt}{1ex}{0pt}%mdk
\noindent\textbf{Theorem~1.} ({\itshape Gidas, Ni, Nirenberg})\mdbr
 假设$u=u(x)$是定义在单位球$B_1\subset \mathbb{R}^n$, $n\geq 2$, 上的正函数, 如果它满足%mdk

\[\begin{cases}
-\Delta u=f(u),&x\in B_1\\
u=0,&x\in\partial B_1,
\end{cases}
\]%mdk%mdk

其中$f: [0,+\infty)\to\mathbb{R}$是Lipschitz连续的, 且$u\in C^2(B_1)\cap C^0(\overline{B_1})$, 则%mdk

\begin{enumerate}[noitemsep,topsep=\mdcompacttopsep]%mdk

\item$u$是镜像对称的. 即对任意的$x,y\in B_1$, 如果$|x|=|y|$, 那么$u(x)=u(y)$;%mdk

\item对任意的$0<r<1$, 有$u'(r)<0$. 即存在$\phi(r)$, 使得对任意的$x\in B_1$, 有$u(x)=\phi(|x|)$, 且对任意的$0<r<1$有$\phi'(r)<0$.%mdk
%mdk
\end{enumerate}%mdk%mdk
\end{mdbmarginx}%mdk

\noindent\textbf{Remark}.
  * 这一结果首先是由B. Gidas, W. Ni, L. Nirenberg\textbackslash{}cite\{GidasNiNirenberg1979Symmetry\}得到的.%mdk%mdk

\noindent[bib]%mdk%mdk


\end{document}
